\documentclass{beamer}

\usepackage[french]{babel}
\usepackage[utf8]{inputenc}
%\usepackage[T1]{fontenc}
\usepackage[french, ruled]{algorithm2e}

\usepackage{graphicx}

\begin{document}

\section{Introduction}

\begin{frame}
  \frametitle{Partage du travail}

\end{frame}

\section{Tours de Hanoï}

\subsection{Implémentation minimale}

\begin{frame}
  \frametitle{Implémentation minimale}

\end{frame}

\subsection{Extensions}

\begin{frame}
  \frametitle{Affichage}
  \includegraphics[width=\columnwidth]{img/hanoi_display.png}
\end{frame}

\begin{frame}
  \begin{center}
    \resizebox{\columnwidth}{!}{
      \begin{tabular}{ | c || c || c | }
        1 & 4 & ~ \\
        \hline
        3 & 6 & ~ \\
        \hline
        5 & 7 & 2 \\
        \hline
      \end{tabular}
    }
  \end{center}
\end{frame}

\begin{frame}
  \frametitle{Tours de Hanoï et plusieurs piques}
  \begin{algorithm}[H]
    \caption{Frame-Stewart($n$, $o$, $d$, $P$)}
    \Entree{$n$ nombre de disques à déplacer, $o$ origine, $d$ destination,
      $P$ piques libres}
    Déplacer $\mathrm{k}(n)$ disques de $o$ à un pique de $P$ \\
    Déplacer $n - \mathrm{k}(n)$ disques de $o$ à $d$ \\
    Déplacer les $\mathrm{k}(n)$ disques a $d$
  \end{algorithm}
\end{frame}

\section{Pavage de Penrose}

\subsection{Implémentation minimal}
\begin{frame}
  \frametitle{Implémentation minimal}
  \begin{columns}
    \column{0.5\columnwidth}
      \includegraphics[width=\columnwidth]{img/acute.pdf}
    \column{0.5\columnwidth}
      \includegraphics[width=\columnwidth]{img/obtuse.pdf}
  \end{columns}
\end{frame}

\begin{frame}
  \begin{columns}
    \column{0.5\columnwidth}
      \includegraphics[width=\columnwidth]{img/obtuse_cut.pdf}
    \column{0.5\columnwidth}
      \includegraphics[width=\columnwidth]{img/acute_cut.pdf}
  \end{columns}
\end{frame}

\begin{frame}
  \begin{center}
    \includegraphics[width=\columnwidth]{img/obtuse_cut_formula.pdf}
    \[
      p_3 = p_1 + \frac{1}{1 + \varphi} (p_2 - p_1)
    \]
  \end{center}
\end{frame}
    
\subsection{Extensions}

\begin{frame}
  \frametitle{Un beau pavage}

\end{frame}

\end{document}
